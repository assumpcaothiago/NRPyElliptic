% *======================================================================*
%  Cactus Thorn template for ThornGuide documentation
%  Author: Ian Kelley
%  Date: Sun Jun 02, 2002
%  $Header$
%
%  Thorn documentation in the latex file doc/documentation.tex
%  will be included in ThornGuides built with the Cactus make system.
%  The scripts employed by the make system automatically include
%  pages about variables, parameters and scheduling parsed from the
%  relevant thorn CCL files.
%
%  This template contains guidelines which help to assure that your
%  documentation will be correctly added to ThornGuides. More
%  information is available in the Cactus UsersGuide.
%
%  Guidelines:
%   - Do not change anything before the line
%       % START CACTUS THORNGUIDE",
%     except for filling in the title, author, date, etc. fields.
%        - Each of these fields should only be on ONE line.
%        - Author names should be separated with a \\ or a comma.
%   - You can define your own macros, but they must appear after
%     the START CACTUS THORNGUIDE line, and must not redefine standard
%     latex commands.
%   - To avoid name clashes with other thorns, 'labels', 'citations',
%     'references', and 'image' names should conform to the following
%     convention:
%       ARRANGEMENT_THORN_LABEL
%     For example, an image wave.eps in the arrangement CactusWave and
%     thorn WaveToyC should be renamed to CactusWave_WaveToyC_wave.eps
%   - Graphics should only be included using the graphicx package.
%     More specifically, with the "\includegraphics" command.  Do
%     not specify any graphic file extensions in your .tex file. This
%     will allow us to create a PDF version of the ThornGuide
%     via pdflatex.
%   - References should be included with the latex "\bibitem" command.
%   - Use \begin{abstract}...\end{abstract} instead of \abstract{...}
%   - Do not use \appendix, instead include any appendices you need as
%     standard sections.
%   - For the benefit of our Perl scripts, and for future extensions,
%     please use simple latex.
%
% *======================================================================*
%
% Example of including a graphic image:
%    \begin{figure}[ht]
% 	\begin{center}
%    	   \includegraphics[width=6cm]{MyArrangement_MyThorn_MyFigure}
% 	\end{center}
% 	\caption{Illustration of this and that}
% 	\label{MyArrangement_MyThorn_MyLabel}
%    \end{figure}
%
% Example of using a label:
%   \label{MyArrangement_MyThorn_MyLabel}
%
% Example of a citation:
%    \cite{MyArrangement_MyThorn_Author99}
%
% Example of including a reference
%   \bibitem{MyArrangement_MyThorn_Author99}
%   {J. Author, {\em The Title of the Book, Journal, or periodical}, 1 (1999),
%   1--16. {\tt http://www.nowhere.com/}}
%
% *======================================================================*

% If you are using CVS use this line to give version information
% $Header$

\documentclass{article}

% Use the Cactus ThornGuide style file
% (Automatically used from Cactus distribution, if you have a
%  thorn without the Cactus Flesh download this from the Cactus
%  homepage at www.cactuscode.org)
\usepackage{../../../../doc/latex/cactus}
\usepackage{xspace}
\newcommand{\nrpyell}{\texttt{NRPyEllipticET}\xspace}

\begin{document}

% The author of the documentation
\author{Leo~R.~Werneck \textless wernecklr@gmail.com\textgreater, \\
  Thiago~Assumpção \textless assumpcaothiago@gmail.com\textgreater,\\
  Zach~Etienne \textless zachetie@gmail.com\textgreater,\\
  Terrence~Pierre~Jacques \textless tp0052@mix.wvu.edu\textgreater}

% The title of the document (not necessarily the name of the Thorn)
\title{\nrpyell}

% the date your document was last changed, if your document is in CVS,
% please use:
%    \date{$ $Date: 2004-01-07 12:12:39 -0800 (Wed, 07 Jan 2004) $ $}
\date{August, 2022}

\maketitle

% Do not delete next line
% START CACTUS THORNGUIDE

% Add all definitions used in this documentation here
%   \def\mydef etc

\begin{abstract}
  \noindent This thorn generates numerical relativity initial data
  by solving the constraints of general relativity using the
  hyperbolic relaxation method.
\end{abstract}


\section{Introduction}

The hyperbolic relaxation method of~\cite{ruter:2018} converts an elliptic
problem into a hyperbolic one. The prototypical elliptic problem---Poisson's
equation---reads
%%
\begin{equation}
  \nabla^{2}\vec{u} = \vec{\rho}\;,
\end{equation}
%%
where $\vec{u}$ is a set of unknowns and $\vec{\rho}$ are sources. One introduces
a \emph{relaxation} time $t$ and replaces this elliptic equation with
%%
\begin{equation}
  \partial_{t}^{2}\vec{u} + \eta\partial_{t}\vec{u} = c^{2}\left(\nabla^{2}\vec{u} - \vec{\rho}\right)\;,
\end{equation}
%%
where $\eta$ is a damping parameter with units of inverse time and $c$ is
the speed of the relaxation waves. One then evolves this hyperbolic equation
until a steady state is reached at $t=t_{\star}$, for which
%%
\begin{equation}
  \left.\partial_{t}^{2}\vec{u}\right|_{t=t_{\star}} = 0 = \left.\partial_{t}\vec{u}\right|_{t=t_{\star}}\;,
\end{equation}
%%
and thus $\left.\vec{u}\right|_{t=t_{\star}}$ is also a solution to the original
elliptic problem. For more details of how this method is implemented in \nrpyell,
please see~\cite{assumpcao:2022}.


\section{Initial data types}
\subsection{Conformally flat binary black hole initial data}

As described in~\cite{assumpcao:2022}, \nrpyell solves the Hamiltonian
constraint using the conformal transverse-traceless (CTT) decomposition, i.e.,
%%
\begin{equation}
  \hat\nabla^{2}u + \frac{1}{8}\tilde{A}_{ij}\tilde{A}^{ij}\left(\psi_{\rm singular}+u\right)^{-7} = 0\;,
\end{equation}
%%
where $\hat\nabla_{i}$ is the covariant derivative compatible with the flat
spatial metric $\hat\gamma_{ij}$, $\hat\nabla^{2} = \hat\nabla_{i}\hat\nabla^{i}$,
$\tilde{A}_{ij} = \psi^{-4}A_{ij}$, where $A_{ij}$ is the traceless part of the
extrinsic curvature, and $\psi$ is the conformal factor, given by
%%
\begin{equation}
  \psi = \psi_{\rm singular} + u \equiv 1 + \sum_{n=1}^{N_{p}}\frac{m_{n}}{2|\vec{x}_{n}|} + u\;,
\end{equation}
%%
where $m_{n}$ are $\vec{x}_{n}$ are the bare mass and position vector of the
$n$\textsuperscript{th} puncture and $u$ is a to-be-determined non-singular
piece of the conformal factor.

We thus solve the Hamiltonian constraint using the hyperbolic relaxation method,
i.e.,
%%
\begin{align}
  \partial_{t}u &= v - \eta u\;,\\
  \partial_{t}v &= c^{2}\left[\hat{\nabla}^{2}u + \frac{1}{8}\tilde{A}_{ij}\tilde{A}^{ij}\left(\psi_{\rm singular} + u\right)\right]\;,
\end{align}
%%
where the first equation defines $v$.

\begin{thebibliography}{9}
  \bibitem{ruter:2018} Rüter,~H.~R., Hildich,~D., Bugner,~M., and Brügmann,~B., Phys. Rev. D \textbf{98}, 084044, 2018 (arXiv: \href{https://arxiv.org/abs/1708.07358}{1708.07358}).
  \bibitem{assumpcao:2022} Assumpção,~T., Werneck,~L.~R., Etienne,~Z.~B., and Pierre Jacques,~T., Phys. Rev. D \textbf{105}, 104037, 2022 (arXiv: \href{https://arxiv.org/abs/2111.02424}{2111.02424}).
\end{thebibliography}

% Do not delete next line
% END CACTUS THORNGUIDE

\end{document}
